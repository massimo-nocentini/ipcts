
\chapter{Examples}

The following sections has been adapted from \cite{doets:eijck}.

\section{ones}

Consider the following Haskell definition for the \emph{constant}
stream of ones:

\inputminted[
    mathescape,
    %linenos,
    %numbersep=5pt,
    %gobble=2,
    %frame=lines,
    %framesep=2mm,
    breaklines,
    firstline=14,
    lastline=14
    ]{haskell}{chapters/code/COR.hs}

The very first one can be taken by definition directly; on the other
hand, asking for the second and third ones need one and two unfolding steps.
Let $\triangleq$ be the relation that allows us to do the unfolding of an equation:

\begin{minted}[escapeinside=||]{haskell}
ones = 1 : ones 
     = 1 : 1 : ones
     = 1 : 1 : 1 : ones
\end{minted}

Have a check:
\begin{minted}[escapeinside=||]{haskell}
*COR> take 20 ones 
[1,1,1,1,1,1,1,1,1,1,1,1,1,1,1,1,1,1,1,1]
\end{minted}

\section{nats}

Now consider the stream of natural numbers:
\inputminted[
    mathescape,
    %linenos,
    %numbersep=5pt,
    %gobble=2,
    %frame=lines,
    %framesep=2mm,
    breaklines,
    firstline=16,
    lastline=16
    ]{haskell}{chapters/code/COR.hs}

Taking the first natural, namely $0$, can be done from the definition directly.
Taking the second natural number, namely $1$, needs one unfolding step, according
the definition of function \mintinline{haskell}|map|:
\begin{minted}{haskell}
nats = 0 : map (+1) nats 
     = 0 : map (+1) (0 : map (+1) nats)
     = 0 : (+1) 0 : map (+1) (map (+1) nats)
     = 0 : 1 : map (+1) (map (+1) nats)
\end{minted}
Taking the third natural number, namely $2$, needs one more unfolding step:
\begin{minted}[escapeinside=||]{haskell}
     = 0 : 1 : map (+1) (map (+1) (0 : map (+1) nats))
     = 0 : 1 : map (+1) ((+1) 0 : map (+1) (map (+1) nats))
     = 0 : 1 : map (+1) (1 : map (+1) (map (+1) nats))
     = 0 : 1 : (+1) 1 : map (+1) (map (+1) (map (+1) nats))
     = 0 : 1 : 2 : map (+1) (map (+1) (map (+1) nats))
\end{minted}
Again, taking the fourth natural number, namely $3$, proceed as before:
\begin{minted}[escapeinside=||]{haskell}
     = 0 : 1 : 2 : map (+1) (map (+1) (map (+1) (0 : map (+1) nats)))
     = 0 : 1 : 2 : map (+1) (map (+1) ((+1) 0 : map (+1) (map (+1) nats)))
     = 0 : 1 : 2 : map (+1) (map (+1) (1 : map (+1) (map (+1) nats)))
     = 0 : 1 : 2 : map (+1) ((+1) 1 : map (+1) (map (+1) (map (+1) nats)))
     = 0 : 1 : 2 : map (+1) (2 : map (+1) (map (+1) (map (+1) nats)))
     = 0 : 1 : 2 : (+1) 2 : map (+1) (map (+1) (map (+1) (map (+1) nats)))
     = 0 : 1 : 2 : 3 : map (+1) (map (+1) (map (+1) (map (+1) nats)))
\end{minted}

Have a check:
\begin{minted}[escapeinside=||]{haskell}
*COR> take 20 theNats
[0,1,2,3,4,5,6,7,8,9,10,11,12,13,14,15,16,17,18,19]
\end{minted}

\section{theFibs}

Now consider the stream of Fibonacci numbers:
\inputminted[
    mathescape,
    %linenos,
    %numbersep=5pt,
    %gobble=2,
    %frame=lines,
    %framesep=2mm,
    breaklines,
    firstline=26,
    lastline=26
    ]{haskell}{chapters/code/COR.hs}

\begin{minted}[escapeinside=||]{haskell}
 theFibs = 0 : 1 : zipWith (+) theFibs (tail theFibs)
         = 0 : 1 : zipWith (+) (0 : 1 : zipWith (+) theFibs (tail theFibs)) 
            (tail (0 : 1 : zipWith (+) theFibs (tail theFibs)))
         = 0 : 1 : zipWith (+) (0 : 1 : zipWith (+) theFibs (tail theFibs)) 
            (1 : zipWith (+) theFibs (tail theFibs))
         = 0 : 1 : (+) 0 1 : zipWith (+) (1 : zipWith (+) theFibs (tail theFibs)) 
            (zipWith (+) theFibs (tail theFibs))
         = 0 : 1 : 1 : zipWith (+) (1 : zipWith (+) theFibs (tail theFibs)) 
            (zipWith (+) theFibs (tail theFibs))
         = 0 : 1 : 1 : zipWith (+) (1 : zipWith (+) theFibs (tail theFibs)) 
            (1 : zipWith (+) (1 : zipWith (+) theFibs (tail theFibs)) 
                (zipWith (+) theFibs (tail theFibs)))
         = 0 : 1 : 1 : (+) 1 1 : zipWith (+) (zipWith (+) theFibs (tail theFibs)) 
            (zipWith (+) (1 : zipWith (+) theFibs (tail theFibs)) 
                (zipWith (+) theFibs (tail theFibs)))
         = 0 : 1 : 1 : 2 : zipWith (+) (zipWith (+) theFibs (tail theFibs)) 
            (zipWith (+) (1 : zipWith (+) theFibs (tail theFibs)) 
                (zipWith (+) theFibs (tail theFibs)))
         = 0 : 1 : 1 : 2 : zipWith (+) (zipWith (+) theFibs (tail theFibs)) 
            (zipWith (+) (1 : zipWith (+) theFibs (tail theFibs)) 
                (1 : zipWith (+) (1 : zipWith (+) theFibs (tail theFibs)) 
                    (zipWith (+) theFibs (tail theFibs))))
         = 0 : 1 : 1 : 2 : zipWith (+) (zipWith (+) theFibs (tail theFibs)) 
            ((+) 1 1 : zipWith (+) (zipWith (+) theFibs (tail theFibs)) 
                (zipWith (+) (1 : zipWith (+) theFibs (tail theFibs)) 
                    (zipWith (+) theFibs (tail theFibs))))
         = 0 : 1 : 1 : 2 : zipWith (+) (zipWith (+) theFibs (tail theFibs)) 
            (2 : zipWith (+) (zipWith (+) theFibs (tail theFibs)) 
                (zipWith (+) (1 : zipWith (+) theFibs (tail theFibs)) 
                    (zipWith (+) theFibs (tail theFibs))))
         = 0 : 1 : 1 : 2 : zipWith (+) 
            (1 : zipWith (+) (1 : zipWith (+) theFibs (tail theFibs)) 
                (zipWith (+) theFibs (tail theFibs)))
            (2 : zipWith (+) (zipWith (+) theFibs (tail theFibs)) 
                (zipWith (+) (1 : zipWith (+) theFibs (tail theFibs)) 
                    (zipWith (+) theFibs (tail theFibs))))
         = 0 : 1 : 1 : 2 : (+) 1 2 : zipWith (+) 
            (zipWith (+) (1 : zipWith (+) theFibs (tail theFibs)) 
                (zipWith (+) theFibs (tail theFibs)))
            (zipWith (+) (zipWith (+) theFibs (tail theFibs)) 
                (zipWith (+) (1 : zipWith (+) theFibs (tail theFibs)) 
                    (zipWith (+) theFibs (tail theFibs))))
         = 0 : 1 : 1 : 2 : 3 : zipWith (+) 
            (zipWith (+) (1 : zipWith (+) theFibs (tail theFibs)) 
                (zipWith (+) theFibs (tail theFibs)))
            (zipWith (+) (zipWith (+) theFibs (tail theFibs)) 
                (zipWith (+) (1 : zipWith (+) theFibs (tail theFibs)) 
                    (zipWith (+) theFibs (tail theFibs))))
\end{minted}

Have a check:
\begin{minted}[escapeinside=||]{haskell}
*COR> take 20 theFibs 
[0,1,1,2,3,5,8,13,21,34,55,89,144,233,377,610,987,1597,2584,4181]
\end{minted}


\section{Eratosthenes' sieve}

\subsection{A solution using an helper \emph{marking} function}

Now consider a first version of Eratosthenes' sieve:
\inputminted[
    mathescape,
    %linenos,
    %numbersep=5pt,
    %gobble=2,
    %frame=lines,
    %framesep=2mm,
    breaklines,
    firstline=30,
    lastline=34
    ]{haskell}{chapters/code/COR.hs}
Previous definition is quite different from \mintinline{haskell}|ones|,
\mintinline{haskell}|nats| and \mintinline{haskell}|theFibs| since it
consumes a stream of integer, namely the stream we need to process
to get prime numbers out of it; in other words, it isn't self contained.

See it in action:
\begin{minted}[escapeinside=||]{haskell}
*COR> take 100 (sieve [2..])
[2,3,5,7,11,13,17,19,23,29,31,37,41,43,47,53,59,61,67,71,73,79,83,89,97,
101,103,107,109,113,127,131,137,139,149,151,157,163,167,173,179,181,191,
193,197,199,211,223,227,229,233,239,241,251,257,263,269,271,277,281,283,
293,307,311,313,317,331,337,347,349,353,359,367,373,379,383,389,397,401,
409,419,421,431,433,439,443,449,457,461,463,467,479,487,491,499,503,509,
521,523,541]
\end{minted}
Instantiating the definition we have:
\begin{minted}{haskell}
sieve (2 : [3..]) = 2 : sieve (mark [3..] 1 2) where
    mark (3:[4..]) 1 2 = 3 : (mark [4..] (1+1) 2)
                       = 3 : (mark [4..] 2 2)
\end{minted}
therefore the first prime, namely $2$, can be extracted by pattern matching
on \mintinline{haskell}|sieve (2 : [3..]) = 2 : ss where ss = sieve (mark [3..] 1 2)|
(we have introduced variable \mintinline{haskell}|ss| in order to emphasize
that the value it is bound to \emph{is not} necessary for computing the first prime).

Asking for the next prime, namely $3$, requires to actually evaluate
\mintinline{haskell}|sieve (mark [3..] 1 2)|. By \mintinline{haskell}{sieve} definition we have
to match \mintinline{haskell}{(mark [3..] 1 2)} to pattern \mintinline{haskell}{n : xs}, therefore we've to
find \mintinline{haskell}{n}, leaving \mintinline{haskell}{xs} lazy evaluated:
\begin{minted}{haskell}
sieve (2 : [3..]) = 2 : sieve (3 : (mark [4..] 2 2)) 
                  = 2 : 3 : sieve (mark (mark [4..] 2 2) 1 3) where
    mark (0 : (mark [5..] 1 2)) 1 3 = 0 : (mark (mark [5..] 1 2) (1+1) 3)
                                    = 0 : (mark (mark [5..] 1 2) 2 3)
\end{minted}
where evaluation of \mintinline{haskell}|(mark [4..] 2 2)| produces
\mintinline{haskell}|(0 : (mark [5..] 1 2))|, required by pattern matching
in the \mintinline{haskell}|where| environment.

Asking for the next prime, namely $5$, requires to actually evaluate
\mintinline{haskell}|sieve (mark (mark [4..] 2 2) 1 3)|. By \mintinline{haskell}{sieve} definition we have
to match \mintinline{haskell}{(mark (mark [4..] 2 2) 1 3)} to pattern \mintinline{haskell}{n : xs}, therefore we've to
find \mintinline{haskell}{n}, leaving \mintinline{haskell}{xs} lazy evaluated:
\begin{minted}{haskell}
sieve (2 : [3..]) = 2 : 3 : sieve (0 : (mark (mark [5..] 1 2) 2 3)) 
                  = 2 : 3 : sieve (mark (mark [5..] 1 2) 2 3)
\end{minted}
since we're required to find prime \mintinline{haskell}|p| such that \mintinline{haskell}|2:3:p:xs|,
for some list \mintinline{haskell}|xs|, \mintinline{haskell}|(mark (mark [5..] 1 2) 2 3)|
needs to be evaluated by definition of \mintinline{haskell}|sieve|. In turn,
applying rules of \mintinline{haskell}|mark|, \mintinline{haskell}|(mark [5..] 1 2)| needs
to be evaluated too, which yield \mintinline{haskell}|5 : (mark [6..] (1+1) 2) = 5 : (mark [6..] 2 2)|.
Therefore \mintinline{haskell}|(mark (5 : (mark [6..] 2 2) 2 3))| evaluates to
\mintinline{haskell}|(5 : (mark (mark [6..] 2 2) (2+1) 3))|, which is
the same to say \mintinline{haskell}|(5 : (mark (mark [6..] 2 2) 3 3))|.
Now we can approach \mintinline{haskell}|sieve| unfolding:
\begin{minted}{haskell}
sieve (2 : [3..]) = 2 : 3 : sieve (5 : (mark (mark [6..] 2 2) 3 3))
                  = 2 : 3 : 5 : sieve (mark (mark (mark [6..] 2 2) 3 3) 1 5) where
    mark (mark (mark [6..] 2 2) 3 3) 1 5
        = mark (mark (0 : (mark [7..] 1 2)) 3 3) 1 5  
        = mark (0 : (mark (mark [7..] 1 2) 1 3)) 1 5
            = 0 : (mark (mark (mark [7..] 1 2) 1 3) (1+1) 5)
            = 0 : (mark (mark (mark [7..] 1 2) 1 3) 2 5)
\end{minted}

Doing one more step, we would like to know the next prime:
\begin{minted}{haskell}
sieve (2 : [3..]) = 2 : 3 : 5 : sieve (0 : (mark (mark (mark [7..] 1 2) 1 3) 2 5))
                  = 2 : 3 : 5 : sieve (mark (mark (mark [7..] 1 2) 1 3) 2 5)
\end{minted}
Therefore we must repeatedly evaluate \mintinline{haskell}|(mark (mark (mark [7..] 1 2) 1 3) 2 5)|:
\begin{minted}{haskell}
(mark (mark (mark [7..] 1 2) 1 3) 2 5)
    = (mark (mark (7 : (mark [8..] (1+1) 2)) 1 3) 2 5)
    = (mark (mark (7 : (mark [8..] 2 2)) 1 3) 2 5)
    = (mark (7 : (mark (mark [8..] 2 2) (1+1) 3)) 2 5)
    = (mark (7 : (mark (mark [8..] 2 2) 2 3)) 2 5)
    = 7 : (mark (mark (mark [8..] 2 2) 2 3) (2+1) 5)
    = 7 : (mark (mark (mark [8..] 2 2) 2 3) 3 5)
\end{minted}
Now we can approach \mintinline{haskell}|sieve| unfolding:
\begin{minted}{haskell}
sieve (2 : [3..]) = 2 : 3 : 5 : sieve (7 : (mark (mark (mark [8..] 2 2) 2 3) 3 5))
                  = 2 : 3 : 5 : 7 : 
                      sieve (mark (mark (mark (mark [8..] 2 2) 2 3) 3 5) 1 7) where
    mark (mark (mark (mark [8..] 2 2) 2 3) 3 5) 1 7
        = mark (mark (mark (0 : (mark [9..] 1 2)) 2 3) 3 5) 1 7
        = mark (mark (0 : (mark (mark [9..] 1 2) (2+1) 3)) 3 5) 1 7
        = mark (mark (0 : (mark (mark [9..] 1 2) 3 3)) 3 5) 1 7
        = mark (0 : (mark (mark (mark [9..] 1 2) 3 3) (3+1) 5)) 1 7
        = mark (0 : (mark (mark (mark [9..] 1 2) 3 3) 4 5)) 1 7
        = 0 : (mark (mark (mark (mark [9..] 1 2) 3 3) 4 5) (1+1) 7)
        = 0 : (mark (mark (mark (mark [9..] 1 2) 3 3) 4 5) 2 7)
\end{minted}
Doing one more step, we would like to know the next prime:
\begin{minted}{haskell}
sieve (2 : [3..]) = 2 : 3 : 5 : 7 : 
                      sieve (0 : (mark (mark (mark (mark [9..] 1 2) 3 3) 4 5) 2 7))
                    = 2 : 3 : 5 : 7 : 
                      sieve (mark (mark (mark (mark [9..] 1 2) 3 3) 4 5) 2 7)
\end{minted}
Therefore we must repeatedly evaluate \mintinline{haskell}|(mark (mark (mark (mark [9..] 1 2) 3 3) 4 5) 2 7)|:
\begin{minted}{haskell}
(mark (mark (mark (mark [9..] 1 2) 3 3) 4 5) 2 7)
    = (mark (mark (mark (9 : (mark [10..] (1+1) 2)) 3 3) 4 5) 2 7)
    = (mark (mark (mark (9 : (mark [10..] 2 2)) 3 3) 4 5) 2 7)
    = (mark (mark (0 : (mark (mark [10..] 2 2) 1 3)) 4 5) 2 7)
    = (mark (0 : (mark (mark (mark [10..] 2 2) 1 3) (4+1) 5)) 2 7)
    = (mark (0 : (mark (mark (mark [10..] 2 2) 1 3) 5 5)) 2 7)
    = 0 : (mark (mark (mark (mark [10..] 2 2) 1 3) 5 5) (2+1) 7)
    = 0 : (mark (mark (mark (mark [10..] 2 2) 1 3) 5 5) 3 7)
\end{minted}
Now we can approach \mintinline{haskell}|sieve| unfolding:
\begin{minted}{haskell}
sieve (2 : [3..]) = 2 : 3 : 5 : 7 : 
                      sieve (0 : (mark (mark (mark (mark [10..] 2 2) 1 3) 5 5) 3 7)) 
                  = 2 : 3 : 5 : 7 : 
                      sieve (mark (mark (mark (mark [10..] 2 2) 1 3) 5 5) 3 7) where
    mark (mark (mark (mark [10..] 2 2) 1 3) 5 5) 3 7
        = mark (mark (mark (0 : mark [11..] 1 2) 1 3) 5 5) 3 7
        = mark (mark (0 : (mark (mark [11..] 1 2) (1+1) 3)) 5 5) 3 7
        = mark (mark (0 : (mark (mark [11..] 1 2) 2 3)) 5 5) 3 7
        = mark (0 : (mark (mark (mark [11..] 1 2) 2 3) 1 5)) 3 7
        = 0 : (mark (mark (mark (mark [11..] 1 2) 2 3) 1 5) (3+1) 7)
        = 0 : (mark (mark (mark (mark [11..] 1 2) 2 3) 1 5) 4 7)
\end{minted}
Doing one more step, we would like to know the next prime:
\begin{minted}{haskell}
sieve (2 : [3..]) = 2 : 3 : 5 : 7 : 
                      sieve (0 : (mark (mark (mark (mark [11..] 1 2) 2 3) 1 5) 4 7))
                    = 2 : 3 : 5 : 7 : 
                      sieve (mark (mark (mark (mark [11..] 1 2) 2 3) 1 5) 4 7)
\end{minted}
Therefore we must repeatedly evaluate \mintinline{haskell}|(mark (mark (mark (mark [11..] 1 2) 2 3) 1 5) 4 7)|:
\begin{minted}{haskell}
(mark (mark (mark (mark [11..] 1 2) 2 3) 1 5) 4 7)
    = (mark (mark (mark (11 : (mark [12..] (1+1) 2)) 2 3) 1 5) 4 7)
    = (mark (mark (mark (11 : (mark [12..] 2 2)) 2 3) 1 5) 4 7)
    = (mark (mark (11 : (mark (mark [12..] 2 2) (2+1) 3)) 1 5) 4 7)
    = (mark (mark (11 : (mark (mark [12..] 2 2) 3 3)) 1 5) 4 7)
    = (mark (11 : (mark (mark (mark [12..] 2 2) 3 3) (1+1) 5)) 4 7)
    = (mark (11 : (mark (mark (mark [12..] 2 2) 3 3) 2 5)) 4 7)
    = 11 : (mark (mark (mark (mark [12..] 2 2) 3 3) 2 5) (4+1) 7)
    = 11 : (mark (mark (mark (mark [12..] 2 2) 3 3) 2 5) 5 7)
\end{minted}
Now we can approach \mintinline{haskell}|sieve| unfolding:
\begin{minted}{haskell}
sieve (2 : [3..]) = 2 : 3 : 5 : 7 : 
                      sieve (11 : (mark (mark (mark (mark [12..] 2 2) 3 3) 2 5) 5 7))
                  = 2 : 3 : 5 : 7 : 11 :
                      sieve (mark (mark (mark (mark (mark [12..] 2 2) 3 3) 2 5) 5 7) 1 11)
    where mark (mark (mark (mark (mark [12..] 2 2) 3 3) 2 5) 5 7) 1 11
            = mark (mark (mark (mark (0 : (mark [13..] 1 2)) 3 3) 2 5) 5 7) 1 11
            = mark (mark (mark (0 : (mark (mark [13..] 1 2) 1 3)) 2 5) 5 7) 1 11
            = mark (mark (0 : (mark (mark (mark [13..] 1 2) 1 3) (2+1) 5)) 5 7) 1 11
            = mark (mark (0 : (mark (mark (mark [13..] 1 2) 1 3) 3 5)) 5 7) 1 11
            = mark (0 : (mark (mark (mark (mark [13..] 1 2) 1 3) 3 5) (5+1) 7)) 1 11
            = mark (0 : (mark (mark (mark (mark [13..] 1 2) 1 3) 3 5) 6 7)) 1 11
            = 0 : (mark (mark (mark (mark (mark [13..] 1 2) 1 3) 3 5) 6 7) (1+1) 11)
            = 0 : (mark (mark (mark (mark (mark [13..] 1 2) 1 3) 3 5) 6 7) 2 11)
\end{minted}

\subsection{A solution using filtering}

The following is another implementation of the Eratosthenes' sieve
which uses filtering:
\inputminted[
    mathescape,
    %linenos,
    %numbersep=5pt,
    %gobble=2,
    %frame=lines,
    %framesep=2mm,
    breaklines,
    firstline=37,
    lastline=38
    ]{haskell}{chapters/code/COR.hs}
See it in action:
\begin{minted}[escapeinside=||]{haskell}
*COR> take 100 (sieve' [2..])
[2,3,5,7,11,13,17,19,23,29,31,37,41,43,47,53,59,61,67,71,73,79,83,89,97,
101,103,107,109,113,127,131,137,139,149,151,157,163,167,173,179,181,191,
193,197,199,211,223,227,229,233,239,241,251,257,263,269,271,277,281,283,
293,307,311,313,317,331,337,347,349,353,359,367,373,379,383,389,397,401,
409,419,421,431,433,439,443,449,457,461,463,467,479,487,491,499,503,509,
521,523,541]
\end{minted}
the same result denoted by \mintinline{haskell}|sieve|.

\section{Doubly linked lists and graphs}

The following sections have been adapted from Coutts slides \cite{coutts}.

\subsection{Doubly linked lists}

The following chunk of code shows how to promote an infinite list
to a doubly linked one:
\inputminted[
    mathescape,
    %linenos,
    %numbersep=5pt,
    %gobble=2,
    %frame=lines,
    %framesep=2mm,
    breaklines,
    firstline=53,
    lastline=72
    ]{haskell}{chapters/code/corecursion.hs}

Have a check:
\begin{minted}[escapeinside=||]{haskell}
*Main> let doubly = to_doubly_linked [1..]
*Main> (next_node . prev_node . next_node . next_node) doubly
3
\end{minted}

\subsection{Graphs}

\subsubsection{Single outgoing connections}

The following chunk of code shows how to build a graph
structure consuming a table specification such that
each node has exactly \emph{one} outgoing connection:
\inputminted[
    %mathescape,
    %linenos,
    %numbersep=5pt,
    %gobble=2,
    %frame=lines,
    %framesep=2mm,
    breaklines,
    firstline=74,
    lastline=82
    ]{haskell}{chapters/code/corecursion.hs}
A table specification looks like this:
\begin{displaymath}
    \begin{array}{c|c|c}
        node & index & ref\\
        \hline
        a & 0 & 1\\
        b & 1 & 0\\
        c & 2 & 4\\
        d & 3 & 4\\
        e & 4 & 2\\
    \end{array}
\end{displaymath}

\subsubsection{Arbitrary outgoing connections}

The following chunk of code shows how to build a graph
structure consuming a table specification:
\inputminted[
    %mathescape,
    %linenos,
    %numbersep=5pt,
    %gobble=2,
    %frame=lines,
    %framesep=2mm,
    breaklines,
    firstline=84,
    lastline=89
    ]{haskell}{chapters/code/corecursion.hs}
A table specification looks like this:
\begin{displaymath}
    \begin{array}{c|c|c}
        node & index & refs\\
        \hline
        a & 0 & \lbrace 1, 4 \rbrace\\
        b & 1 & \lbrace 1, 2, 3 \rbrace\\
        c & 2 & \lbrace \rbrace\\
        d & 3 & \lbrace 0, 4 \rbrace\\
        e & 4 & \lbrace 3, 4 \rbrace\\
    \end{array}
\end{displaymath}
