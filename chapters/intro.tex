
\chapter{Introduction}

Consider the following Haskell definition for the \emph{constant}
stream of ones:

\inputminted[
    mathescape,
    %linenos,
    %numbersep=5pt,
    %gobble=2,
    %frame=lines,
    %framesep=2mm,
    breaklines,
    firstline=14,
    lastline=14
    ]{haskell}{chapters/code/COR.hs}

The very first one can be taken by definition directly; on the other
hand, asking for the second and third ones need one and two unfolding steps.
Let $\triangleq$ be the relation that allows us to do the unfolding of an equation:

\begin{minted}[escapeinside=||]{haskell}
    ones = 1 : ones 
         = 1 : 1 : ones
         = 1 : 1 : 1 : ones
\end{minted}

Now consider the stream of natural numbers:
\inputminted[
    mathescape,
    %linenos,
    %numbersep=5pt,
    %gobble=2,
    %frame=lines,
    %framesep=2mm,
    breaklines,
    firstline=16,
    lastline=16
    ]{haskell}{chapters/code/COR.hs}

Taking the first natural, namely $0$, can be done from the definition directly.
Taking the second natural number, namely $1$, needs one unfolding step, according
the definition of function \mintinline{haskell}|map|:
\begin{minted}{haskell}
    nats = 0 : map (+1) nats 
         = 0 : map (+1) (0 : map (+1) nats)
         = 0 : (+1) 0 : map (+1) (map (+1) nats)
         = 0 : 1 : map (+1) (map (+1) nats)
\end{minted}
Taking the third natural number, namely $2$, needs one more unfolding step:
\begin{minted}[escapeinside=||]{haskell}
         = 0 : 1 : map (+1) (map (+1) (0 : map (+1) nats))
         = 0 : 1 : map (+1) ((+1) 0 : map (+1) (map (+1) nats))
         = 0 : 1 : map (+1) (1 : map (+1) (map (+1) nats))
         = 0 : 1 : (+1) 1 : map (+1) (map (+1) (map (+1) nats))
         = 0 : 1 : 2 : map (+1) (map (+1) (map (+1) nats))
\end{minted}
Again, taking the fourth natural number, namely $3$, proceed as before:
\begin{minted}[escapeinside=||]{haskell}
         = 0 : 1 : 2 : map (+1) (map (+1) (map (+1) (0 : map (+1) nats)))
         = 0 : 1 : 2 : map (+1) (map (+1) ((+1) 0 : map (+1) (map (+1) nats)))
         = 0 : 1 : 2 : map (+1) (map (+1) (1 : map (+1) (map (+1) nats)))
         = 0 : 1 : 2 : map (+1) ((+1) 1 : map (+1) (map (+1) (map (+1) nats)))
         = 0 : 1 : 2 : map (+1) (2 : map (+1) (map (+1) (map (+1) nats)))
         = 0 : 1 : 2 : (+1) 2 : map (+1) (map (+1) (map (+1) (map (+1) nats)))
         = 0 : 1 : 2 : 3 : map (+1) (map (+1) (map (+1) (map (+1) nats)))
\end{minted}

Now consider the stream of Fibonacci numbers:
\inputminted[
    mathescape,
    %linenos,
    %numbersep=5pt,
    %gobble=2,
    %frame=lines,
    %framesep=2mm,
    breaklines,
    firstline=26,
    lastline=26
    ]{haskell}{chapters/code/COR.hs}

\begin{minted}[escapeinside=||]{haskell}
     theFibs = 0 : 1 : zipWith (+) theFibs (tail theFibs)
             = 0 : 1 : zipWith (+) (0 : 1 : zipWith (+) theFibs (tail theFibs)) 
                (tail (0 : 1 : zipWith (+) theFibs (tail theFibs)))
             = 0 : 1 : zipWith (+) (0 : 1 : zipWith (+) theFibs (tail theFibs)) 
                (1 : zipWith (+) theFibs (tail theFibs))
             = 0 : 1 : (+) 0 1 : zipWith (+) (1 : zipWith (+) theFibs (tail theFibs)) 
                (zipWith (+) theFibs (tail theFibs))
             = 0 : 1 : 1 : zipWith (+) (1 : zipWith (+) theFibs (tail theFibs)) 
                (zipWith (+) theFibs (tail theFibs))
             = 0 : 1 : 1 : zipWith (+) (1 : zipWith (+) theFibs (tail theFibs)) 
                (1 : zipWith (+) (1 : zipWith (+) theFibs (tail theFibs)) 
                    (zipWith (+) theFibs (tail theFibs)))
             = 0 : 1 : 1 : (+) 1 1 : zipWith (+) (zipWith (+) theFibs (tail theFibs)) 
                (zipWith (+) (1 : zipWith (+) theFibs (tail theFibs)) 
                    (zipWith (+) theFibs (tail theFibs)))
             = 0 : 1 : 1 : 2 : zipWith (+) (zipWith (+) theFibs (tail theFibs)) 
                (zipWith (+) (1 : zipWith (+) theFibs (tail theFibs)) 
                    (zipWith (+) theFibs (tail theFibs)))
             = 0 : 1 : 1 : 2 : zipWith (+) (zipWith (+) theFibs (tail theFibs)) 
                (zipWith (+) (1 : zipWith (+) theFibs (tail theFibs)) 
                    (1 : zipWith (+) (1 : zipWith (+) theFibs (tail theFibs)) 
                        (zipWith (+) theFibs (tail theFibs))))
             = 0 : 1 : 1 : 2 : zipWith (+) (zipWith (+) theFibs (tail theFibs)) 
                ((+) 1 1 : zipWith (+) (zipWith (+) theFibs (tail theFibs)) 
                    (zipWith (+) (1 : zipWith (+) theFibs (tail theFibs)) 
                        (zipWith (+) theFibs (tail theFibs))))
             = 0 : 1 : 1 : 2 : zipWith (+) (zipWith (+) theFibs (tail theFibs)) 
                (2 : zipWith (+) (zipWith (+) theFibs (tail theFibs)) 
                    (zipWith (+) (1 : zipWith (+) theFibs (tail theFibs)) 
                        (zipWith (+) theFibs (tail theFibs))))
             = 0 : 1 : 1 : 2 : zipWith (+) 
                (1 : zipWith (+) (1 : zipWith (+) theFibs (tail theFibs)) 
                    (zipWith (+) theFibs (tail theFibs)))
                (2 : zipWith (+) (zipWith (+) theFibs (tail theFibs)) 
                    (zipWith (+) (1 : zipWith (+) theFibs (tail theFibs)) 
                        (zipWith (+) theFibs (tail theFibs))))
             = 0 : 1 : 1 : 2 : (+) 1 2 : zipWith (+) 
                (zipWith (+) (1 : zipWith (+) theFibs (tail theFibs)) 
                    (zipWith (+) theFibs (tail theFibs)))
                (zipWith (+) (zipWith (+) theFibs (tail theFibs)) 
                    (zipWith (+) (1 : zipWith (+) theFibs (tail theFibs)) 
                        (zipWith (+) theFibs (tail theFibs))))
             = 0 : 1 : 1 : 2 : 3 : zipWith (+) 
                (zipWith (+) (1 : zipWith (+) theFibs (tail theFibs)) 
                    (zipWith (+) theFibs (tail theFibs)))
                (zipWith (+) (zipWith (+) theFibs (tail theFibs)) 
                    (zipWith (+) (1 : zipWith (+) theFibs (tail theFibs)) 
                        (zipWith (+) theFibs (tail theFibs))))
\end{minted}
